\documentclass[a4paper, norsk, 11pt]{report}
\usepackage[latin1]{inputenc}
\usepackage{babel}

\author{Arne Dahl Bjune, Hanne Riise M�hlum, Katarina Hokstad}
\title{TTM4135 Informasjonssikkerhet Rapport}

%\rhead{Topptekst}06.03.2012


\begin{document}
\maketitle
\setlength{\parindent}{0pt}
\setlength{\parskip}{2ex}


\section{Introduction}
Litt introduksjon og s�nn
\section{Experimental}
Her eksprimenterte vi med masse rart
\section{Procedure}
Dette var litt foskjelig
\section{Results}
Funka bra!
\section{Discussion}

\paragraph{Q1} 
Comment on security related issues regarding the cryptographic algorithms used to
generate and sign your groups web server certi?cate (key length, algorithm, etc.).
\par
All the CAs provided by the staff are signed with a SHA1 hash, our ca uses a SHA256 key. Both uses a 2048bit key length. SHA1 is no longer recommended to use on digital signatures by NIST[4] because of an attack by Xiaoyun Wang[1] reducing the probability of a collision from 2 ^ 80 to 2 ^ 63 [2][3]. A key length of 2048 bits is considered secure until 2030 by NIST[5], 2020 by FNISA[6] and BSI[7]. The Durotation of the certificate is set to 365 days which is within the recommendations set by NIST[8].

[1] Finding Collisions in the Full SHA-1Collision Search Attacks on SHA1
	http://people.csail.mit.edu/yiqun/SHA1AttackProceedingVersion.pdf
[2] http://www.schneier.com/blog/archives/2005/08/new_cryptanalyt.html	Droppe?, bruke [3]?
[3] http://csrc.nist.gov/groups/ST/hash/statement.html
[4] http://csrc.nist.gov/groups/ST/toolkit/secure_hashing.html
[5] Trenger bedre ref en dette: http://www.keylength.com/en/
[6] Trenger bedre ref en dette: http://www.keylength.com/en/
[7] Trenger bedre ref en dette: http://www.keylength.com/en/
[8]



\paragraph{Q2} 
Explain what you have achieved through each of these verifcations. What is the name
of the person signing the Apache release?
\par
Pretty Good Privacy (PGP) is a computer program that encrypt and decrypt data. It gives a  confidentiality and authentication service. The first version of this program was a web of trust system.
\par
By verifying the source code whit the PGP program we ensure that the source code is the original code and that no one has tampered with it. The PGP program uses a private key and a public key. The private key is kept by the person wishing to authenticate it self and the public key is given to the people he wants to authenticate him self to. The sender signs a hash whit the private key. When an other person receives the data he can use the public key to check that the person is who he claims to be. 
\par
PGP uses web of trust to ensure that the person signing the Aparch release is who he clams to be. A web of trust is a network were .
\paragraph{Q3}
What are the access permissions to your web server�s con?guration ?les, server certi?cate
and the corresponding private key? Comment on possible attacks to your web server due to
inappropriate ?le permissions
\par
The web server should preferably be run by a user other than the one that owns the configuration files. This is not possible for us in this lab, since we don�t have the ability to create new users. If the web server has the opportunity to write to the configuration files, you can write a new configuration file that opens new security holes if you�re hacking into the web server. Our files have read and write access to the owner of the file, but no rights to the members of the group. However, this has nothing to say in this setup since we�re the only user in the group.
\paragraph{Q4}
Web servers o?ering weak cryptography are subject to several attacks. What kind of attacks are feasible? How did you con?gure your server to prevent such attacks?
\par
By allowing weak ciphers and SSLv2 you are vulnerable to Man In The Middle Attack where the attacker changes the content of the Client Hello package. On our server we are only allowing SSLv3 and TLSv1 protocols with HIGH and MEDIUM ciphers as requried by PCI DSS [9] (Payment Card Industry Data Security Standarad)
\par
[9] https://www.pcisecuritystandards.org/pdfs/navigating_pci_dss_v1-1.pdf

\paragraph{Q5}
What kind of malicious attacks is your web application (PHP) vulnerable to? Describe them brie?y, and point out what countermeasures you have developed in your code to prevent such attacks
\par
Our web application is not vulnerable for sessions, since we�re not using it. Cross site scripting is avoided by using strip_tags in PHP on all user input. We have secured our web application against SQL injections, by using stripslashes and mysqli_real_escape_string commands in PHP on all input to databases. 

\paragraph{Q6}
Describe the security measures you have undertaken to secure your repository, and how did that a?ect the security of your Web Application (Better? Worse?). Elaborate on thepossible further measures, that can prevent certain types of attacks you found possible in the setting you created. Can you discover any vulnerabilities in the other groups� projects? If so, try to mount attacks on other groups!
\par
We�ve secured our subversion repository with htcaccess I, and the password is hashed with SHA. This doesn�t give further security alone, but combined with certificates from the /restricted folder you�ll get two component protection with something you got and something you know, certificate and password respectively. 
\par
Some groups forgot to /secure the folder in the HTTP server, which gave us access to everything without certificates. Others forgot to set verify depth to 2 in apache http.conf. If it�s set to 4 or higher, which was the case for these groups, you can sign your own certificates with your own groups CA with their email as common name to gain access. 
\par
Referanser:
PGP: RFC4880 http://tools.ietf.org/html/rfc4880
X.509: RFC4158 http://tools.ietf.org/html/rfc4158

\section{Conclusion}
Kult fag
\section{Appendices}
Her st�r det masse fin tekst

\section{Litt notater}

\paragraph{Q1:} 
Sertifikatet til v�r webserver er signerert av v�r CA. Den signerer med en SHA256 hash som er mye sikrere enn MD5 (som er er standard), 365 dager varighet p� sertfikat
Key length er 2048 bit, req er ogs� sha256

Key length studca ogs� 2048bit, det st�r noe om anbefalt key lenght ut fra algoritme i foilene bassert p� tall fra NSA, sikkert lurt � sitere det.


\paragraph{Q2:} 
Dette handler om at kildekoden vi laster ned ikke har blitt endret. Sikkert lurt � skrive litt om hvordan PGP (Prety good privacy) fungerer i forhold til CA, er totalt forskjelige tiln�rminger til samme problem


	

Signert av (ser fornuftig ut i koden):
\begin{quotation}
gpg: Signature made Wed 25 Jan 2012 11:27:28 PM CET using RSA key ID 60C5442D
gpg: Good signature from "William A. Rowe, Jr. <wrowe@rowe-clan.net>"
gpg:                 aka "William A. Rowe, Jr. <wrowe@apache.org>"
gpg:                 aka "William A. Rowe, Jr. <wrowe@vmware.com>"
gpg:                 aka "William A. Rowe, Jr. <william.rowe@springsource.com>"
gpg: WARNING: This key is not certified with a trusted signature!
gpg:          There is no indication that the signature belongs to the owner.
Primary key fingerprint: B1B9 6F45 DFBD CCF9 7401  9235 193F 180A B55D 9977
     Subkey fingerprint: 627B E9D7 D7C6 9D30 A2F5  B008 5593 BCA9 60C5 442D
\end{quotation}



\paragraph{Q3:} 
Webserveren b�r helst kj�res av en annen bruker en den som eier konfigurasjonsfilene. Det er ikke mulig for oss i denne laben sidne vi ikke har mulighet til � lage nye brukere. Hvis webserveren har mulighet til � skrive til config filene kan man n�r man hacker webserven skrive en ny config som �pner nye sikkerhetshull. V�re filer har lese og skriverettigheter til den som eier filene men ingen rettigheter til medlem av gruppa (det har ingenting � si siden i dette oppsettet er vi eneste bruker i gruppen).



Q5 Ikke s�rbar for sessions siden vi ikke bruker det. S�rbar for Cross site scripting, ingenting gjort for � fikse det, ikke sikker p� hvordan det funker. 
Sikret mot SQL Injections (detaljer i tex koden):

%\$username = stripslashes(\$username);
%\$password = stripslashes(\$password);
%\$username = mysqli_real_escape_string(\$connection,\$username);
%\$password = mysqli_real_escape_string(\$connection,\$password);

Hvis noen skulle f� tilgang til databasen v�r er passord lagret hashet med SHA256 og brukernavn som salt. Salt beskytter mot at alle passordene i databasen kan knekkes samtidig.





\paragraph{Q6:}  Subversion repository er sikkret med htaccess fil, passord hashet med SHA, gir ikke ytligere sikring alene (passord alene kan vel regnes som mer usikkert enn sertifikater, er ikke sikker) men hvis man kombinerer det med certifikatsikringen i fra /restricted mappa vil man f� to komponent sikring, med noe man har (sertifikat) og noe vet (passord)
\paragraph{Fra config fil}
AuthType Basic
AuthName "Subversion repository"
AuthUserFile /home/gr10/subversion/repository/conf/svn-auth-file
Require valid-user
\paragraph{Sikkerhetshull hos andre grupper:} 

Glemt � blokke /secure mappa i HTTP serveren, da f�r man tilgang til alt uten sertifikater
Glemt � sette verify depth til 2 i apache httpd.conf, hvis den st�r til 4 eller h�yere kan man signere sine egne sertifikater med sitt eget group CA og komme inn.


\end{document}
