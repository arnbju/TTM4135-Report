\documentclass[a4paper, norsk, 11pt]{report}
\usepackage[latin1]{inputenc}
\usepackage{babel}

\author{Arne Dahl Bjune, Hanne Riise M�hlum, Katarina Hokstad}
\title{TTM4135 Informasjonssikkerhet Rapport}

%\rhead{Topptekst}06.03.2012


\begin{document}
\maketitle



\section{Litt notater}

\paragraph{Q1:} 
Sertifikatet til v�r webserver er signerert av v�r CA. Den signerer med en SHA256 hash som er mye sikrere enn MD5 (som er er standard), 365 dager varighet p� sertfikat
Key length er 2048 bit, req er ogs� sha256

Key length studca ogs� 2048bit, det st�r noe om anbefalt key lenght ut fra algoritme i foilene bassert p� tall fra NSA, sikkert lurt � sitere det.


\paragraph{Q2:} 
Dette handler om at kildekoden vi laster ned ikke har blitt endret. Sikkert lurt � skrive litt om hvordan PGP (Prety good privacy) fungerer i forhold til CA, er totalt forskjelige tiln�rminger til samme problem


	

Signert av (ser fornuftig ut i koden):
\begin{quotation}
gpg: Signature made Wed 25 Jan 2012 11:27:28 PM CET using RSA key ID 60C5442D
gpg: Good signature from "William A. Rowe, Jr. <wrowe@rowe-clan.net>"
gpg:                 aka "William A. Rowe, Jr. <wrowe@apache.org>"
gpg:                 aka "William A. Rowe, Jr. <wrowe@vmware.com>"
gpg:                 aka "William A. Rowe, Jr. <william.rowe@springsource.com>"
gpg: WARNING: This key is not certified with a trusted signature!
gpg:          There is no indication that the signature belongs to the owner.
Primary key fingerprint: B1B9 6F45 DFBD CCF9 7401  9235 193F 180A B55D 9977
     Subkey fingerprint: 627B E9D7 D7C6 9D30 A2F5  B008 5593 BCA9 60C5 442D
\end{quotation}



\paragraph{Q3:} 
Webserveren b�r helst kj�res av en annen bruker en den som eier konfigurasjonsfilene. Det er ikke mulig for oss i denne laben sidne vi ikke har mulighet til � lage nye brukere. Hvis webserveren har mulighet til � skrive til config filene kan man n�r man hacker webserven skrive en ny config som �pner nye sikkerhetshull. V�re filer har lese og skriverettigheter til den som eier filene men ingen rettigheter til medlem av gruppa (det har ingenting � si siden i dette oppsettet er vi eneste bruker i gruppen).



Q5 Ikke s�rbar for sessions siden vi ikke bruker det. S�rbar for Cross site scripting, ingenting gjort for � fikse det, ikke sikker p� hvordan det funker. 
Sikret mot SQL Injections (detaljer i tex koden):

%\$username = stripslashes(\$username);
%\$password = stripslashes(\$password);
%\$username = mysqli_real_escape_string(\$connection,\$username);
%\$password = mysqli_real_escape_string(\$connection,\$password);

Hvis noen skulle f� tilgang til databasen v�r er passord lagret hashet med SHA256 og brukernavn som salt. Salt beskytter mot at alle passordene i databasen kan knekkes samtidig.





\paragraph{Q6:}  Subversion repository er sikkret med htaccess fil, passord hashet med SHA, gir ikke ytligere sikring alene (passord alene kan vel regnes som mer usikkert enn sertifikater, er ikke sikker) men hvis man kombinerer det med certifikatsikringen i fra /restricted mappa vil man f� to komponent sikring, med noe man har (sertifikat) og noe vet (passord)
\paragraph{Fra config fil}
AuthType Basic
AuthName "Subversion repository"
AuthUserFile /home/gr10/subversion/repository/conf/svn-auth-file
Require valid-user
\paragraph{Sikkerhetshull hos andre grupper:} 

Glemt � blokke /secure mappa i HTTP serveren, da f�r man tilgang til alt uten sertifikater
Glemt � sette verify depth til 2 i apache httpd.conf, hvis den st�r til 4 eller h�yere kan man signere sine egne sertifikater med sitt eget group CA og komme inn.

\section{Introduction}
Litt introduksjon og s�nn
\section{Experimental}
Her eksprimenterte vi med masse rart
\section{Procedure}
Dette var litt foskjelig
\section{Results}
Funka bra!
\section{Discussion}
Blabladblablablabla
\section{Conclusion}
Kult fag
\section{Appendices}
Her st�r det masse fin tekst


\section{Appendices}
Her st�r det masse fin tekst

\end{document}
